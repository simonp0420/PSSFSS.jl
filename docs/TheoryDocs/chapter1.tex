\chapter{Fundamentals}
\label{chap:fund}
This chapter documents the definitions for fields, potentials, and 
Fourier transforms that are employed in remainder of this document.


We restrict consideration to time-harmonic sources in simple, linear media,
assuming and suppressing
a time dependence of $e^{j\omega t}$.  RMS phasers are used throughout, employing
rationalized MKS units. Thus,
if $V$ is a (complex) phaser voltage, then the corresponding function of time
is $v(t) = \sqrt{2} \Real{V e^{j\omega t}}$.

A right-handed Cartesian  coordinate system is adopted, with $x$, $y$, and 
$z$ axes, and unit vectors $\x$, $\y$, and $\z$.  A point $P = (x,y,z)$ is 
typically identified by the vector $\r = x\x + y\y + z\z$ which measures 
the displacement of $P$ from the origin.

The medium under consideration is characterized by its scalar, complex permittivity
$\epsilon$ [F/m] and its scalar, complex permeability $\mu$ [H/m], 
both of which may be functions 
of position, $\epsilon = \epsilon(\r)$, $\mu = \mu(\r)$. Although these parameters are
both positive for lossless media, in the presence of electric and/or magnetic
losses the imaginary part of $\epsilon$ and/or $\mu$, respectively, is negative.
Thus, $\epsilon$ and $\mu$ both lie in the fourth quadrant (or positive real
axis) of the complex plane.

For convenience we define the medium's intrinsic wavenumber
\begin{equation}
  k = \omega \sqrt{\mu\epsilon} \qquad \text{(fourth quadrant)}
\end{equation}
and intrinsic impedance
\begin{equation}
  \eta = \sqrt{\mu/\epsilon} \qquad \left(\abs{\arg\eta} < \frac{\pi}{4} \right),
\end{equation}
which, of course, vary with position if $\epsilon$ or $\mu$ do.

%%%%%%%%%%%%%%%%%%%%%%%%%%%%%%%%%%%%%%%%%%%%%%%%%%%%%%%%%%%%%%%%%%%%%%%%%%%%%%%%%%%%%%%
\section{Maxwell's Equations and Potentials for Electric Sources}
Under the assumptions listed above, and postulating the existence of only
electric sources, Maxwell's curl equations 
(Ampere's Law and Faraday's Law) take the form
\begin{subequations}
  \begin{align}
    \curl \H &= j\omega\epsilon \E + \J \label{eq:AmpereE} \\
    \curl \E &= -j\omega\mu \H  \label{eq:FaradayE}
  \end{align}
\end{subequations}
where $\E$ is the electric field vector [V/m], $\H$ is the magnetic field
vector [A/m], and $\J$ is the electric current density [A/m$^2$].
When combined with the equation of continuity
\begin{equation}
  \divergence \J + j\omega\qe = 0
\end{equation}
we obtain the divergence relations
\begin{subequations}
  \begin{align}
    \divergence \epsilon\E &= \qe  \label{eq:divE}\\
    \divergence \mu\H       &= 0,
  \end{align}
\end{subequations}
where $\qe$ is the electric charge density,
with units of [C/m$^3$].
The fact that $\mu\H$ is divergenceless leads to 
the introduction of the magnetic vector potential $\A$
having units of [Vs/m]:
\begin{equation}
  \boxed{
  \mu\H = \curl \A.  
  } \label{eq:curlA}
\end{equation}
Substituting \eqref{eq:curlA} into \eqref{eq:FaradayE} we find that
$-\E - j\omega\A$ is curl-free, and so can be written as the gradient
of the so-called electric scalar potential $\Phi$ [V]:
\begin{equation}
  \label{eq:PhiE}
  \boxed{
  \E = -j\omega\A-\gradient\Phi.
  }
\end{equation}
To derive the differential equations for $\A$ and $\Phi$ we
begin with the identity
\begin{equation}
  \curl \curl \A = \curl \mu\H = \mu \curl \H + \gradient\mu \cross \H
\end{equation}
and use Equations \eqref{eq:AmpereE} and \eqref{eq:curlA} to eliminate
$\H$:
\begin{equation}
  \gradient\divergence\A - \vlaplace\A =
  j\omega\epsilon\mu\E + \mu\J + \gradient\mu \cross 
  \left(
    \frac{1}{\mu} \curl\A
  \right).
\end{equation}
Note that we also employed the identity $\curl\curl\A = 
  \gradient\divergence\A - \vlaplace\A$.
Now eliminating $\E$ using \eqref{eq:PhiE} we obtain
\begin{equation}
  \vlaplace\A + k^2 \A + \gradient\mu \cross
  \left(
    \frac{1}{\mu} \curl\A
  \right)
  - \gradient 
  \left[
    \divergence\A + j\omega\mu\epsilon\Phi
  \right]
  = -\mu\J.
\end{equation}
Since the divergence of the magnetic vector potential is as yet unspecified,
we may apply the Lorentz gauge, $\divergence\A = -j\omega\epsilon\mu\Phi$,
and set the quantity in square brackets above to zero:
\begin{equation}
  \boxed{%
  \vlaplace\A + k^2 \A + \frac{\gradient\mu}{\mu} \cross \curl\A
  = -\mu\J.  \label{eq:WaveAinhom}
  }
\end{equation}
Equation~\eqref{eq:WaveAinhom} is the fundamental wave equation for 
the magnetic vector potential under the Lorentz gauge in an inhomogeneous
medium.  

An equation for $\Phi$ is now obtained by employing the
identity $\divergence \epsilon\E = \E\bdot\gradient \epsilon
+ \epsilon \divergence\E$ in \eqref{eq:divE} and then
eliminating $\E$ using \eqref{eq:PhiE}:
\begin{align}
  \qe
  &= \E \bdot \gradient \epsilon + \epsilon \divergence\E \notag \\
  &= -(j\omega\A + \gradient\Phi) \bdot \gradient\epsilon
  - \epsilon \divergence (j\omega\A + \gradient\Phi).
\end{align}
After invoking the Lorentz gauge this can be written as
\begin{equation}
  \boxed{%
  \laplace\Phi + k^2\Phi + \frac{\gradient\Phi \bdot \gradient\epsilon}{\epsilon}
  = 
  -\frac{\qe}{\epsilon} - \frac{j\omega\A \bdot \gradient\epsilon}{\epsilon}.
  } \label{eq:WavePhiinhom}
\end{equation}
Equation~\eqref{eq:WavePhiinhom} is the wave equation for 
the electric scalar potential under the Lorentz gauge in an inhomogeneous
medium.  

\subsection{Piecewise Homogeneous Medium}
Suppose that the spatial domain $U$ of the boundary value problem
for which Maxwell's Equations are to be
solved consists of a disjoint union of a finite number $N$ of
homogeneous regions $U_i$, as in the case of a stratified medium:
\begin{equation}
  U = \union_{i=1}^{N} U_i,
\end{equation}
and suppose that the permittivity and permeability of the $i$th region
are the constants $\epsilon_i$ and $\mu_i$, respectively, with corresponding
wavenumber $k_i$.  Then, for
points within the $i$th medium, the terms involving the gradient of
the permittivity and permeability are zero, and the 
potentials within the $i$th region are solutions to 
\begin{subequations}
  \begin{gather}
    \vlaplace\A^{(i)} + k_i^2 \A^{(i)}  = -\mu_i\J, \\
    \laplace\Phi^{(i)} + k_i^2 \Phi^{(i)}  = -\qe/\epsilon_i.
  \end{gather}
\end{subequations}
If the $i$th region contains no sources, then the potentials
in that region are solutions to the Helmholtz equation:
\begin{subequations}
  \begin{gather}
    \vlaplace\A^{(i)} + k_i^2 \A^{(i)}  = \0, \\
    \laplace\Phi^{(i)} + k_i^2 \Phi^{(i)}  = 0.
  \end{gather}
\end{subequations}

%%%%%%%%%%%%%%%%%%%%%%%%%%%%%%%%%%%%%%%%%%%%%%%%%%%%%%%%%%%%%%%%%%%%%%%%%%%%%%%%%%%%%%%
\section{Maxwell's Equations for Magnetic Sources}
Under the assumptions listed in the introduction, and postulating the existence of only
magnetic sources, Maxwell's curl equations 
(Ampere's Law and Faraday's Law) take the form
\begin{subequations}
  \begin{align}
    \curl \H &= j\omega\epsilon \E \label{eq:AmpereM} \\
    \curl \E &= -j\omega\mu \H - \M \label{eq:FaradayM}
  \end{align}
\end{subequations}
where $\E$ is the electric field vector [V/m], $\H$ is the magnetic field
vector [A/m], and $\M$ is the magnetic current density [V/m$^2$].
When combined with the equation of continuity
\begin{equation}
  \divergence \M + j\omega\qm = 0
\end{equation}
we obtain the divergence relations
\begin{subequations}
  \begin{align}
    \divergence \epsilon\E &= 0  \label{eq:divEm}\\
    \divergence \mu\H &= \qm,    \label{eq:divHm}
  \end{align}
\end{subequations}
where $\qm$ is the magnetic charge density,
with units of [Wb/m$^3$].
The fact that $\epsilon\E$ is divergenceless leads to 
the introduction of the electric vector potential $\F$
having units of [As/m]:
\begin{equation}
  \boxed{
  \epsilon\E = \curl \F.  
  } \label{eq:curlF}
\end{equation}
Substituting \eqref{eq:curlF} into \eqref{eq:AmpereM} we find that
$j\omega\F - \H$ is curl-free, and so can be written as the gradient
of the so-called magnetic scalar potential $\Psi$ [A]:
\begin{equation}
  \label{eq:PsiM}
  \boxed{
  \H =  j\omega\F - \gradient\Psi.
  }
\end{equation}
To derive the differential equations for $\F$ and $\Psi$ we
begin with the identity
\begin{equation}
  \curl \curl \F = \curl \epsilon\E = \epsilon \curl \E + \gradient\epsilon \cross \E
\end{equation}
and use Equations \eqref{eq:FaradayM} and \eqref{eq:curlF} to eliminate
$\E$:
\begin{equation}
  \gradient\divergence\F - \vlaplace\F =
  -j\omega\epsilon\mu\H - \epsilon\M + \gradient\epsilon \cross 
  \left(
    \frac{1}{\epsilon} \curl\F
  \right).
\end{equation}
Note that we also employed the identity $\curl\curl\F = 
  \gradient\divergence\F - \vlaplace\F$.
Now eliminating $\H$ using \eqref{eq:PsiM} we obtain
\begin{equation}
  \vlaplace\F + k^2 \F + \gradient\epsilon \cross
  \left(
    \frac{1}{\epsilon} \curl\F
  \right)
  - \gradient 
  \left[
    \divergence\F - j\omega\mu\epsilon\Psi
  \right]
  = \epsilon\M.
\end{equation}
Since the divergence of the electric vector potential is as yet unspecified,
we may apply the Lorentz gauge, $\divergence\F = j\omega\epsilon\mu\Psi$,
and set the quantity in square brackets above to zero:
\begin{equation}
  \boxed{%
  \vlaplace\F + k^2 \F + \frac{\gradient\epsilon}{\epsilon} \cross
  \curl\F
  = \epsilon\M.
  \label{eq:WaveFinhom}
  }
\end{equation}
Equation~\eqref{eq:WaveFinhom} is the fundamental wave equation for 
the electric vector potential under the Lorentz gauge in an inhomogeneous
medium.  

An equation for $\Psi$ is now obtained by employing the
identity $\divergence \mu\H = \H\bdot\gradient \mu
+ \mu \divergence\H$ in \eqref{eq:divHm} and then
eliminating $\H$ using \eqref{eq:PsiM}:
\begin{align}
  \qm
  &= \H \bdot \gradient \mu + \mu \divergence\H \notag \\
  &= (j\omega\F - \gradient\Psi) \bdot \gradient\mu
  + \mu \divergence (j\omega\F - \gradient\Psi).
\end{align}
After invoking the Lorentz gauge this can be written as
\begin{equation}
  \boxed{%
  \laplace\Psi + k^2\Psi + \frac{\gradient\Psi \bdot \gradient\mu}{\mu}
  = 
  -\frac{\qm}{\mu} + \frac{j\omega\F \bdot \gradient\mu}{\mu}.
  } \label{eq:WavePsiinhom}
\end{equation}
Equation~\eqref{eq:WavePsiinhom} is the wave equation for 
the electric scalar potential under the Lorentz gauge in an inhomogeneous
medium.  

\subsection{Piecewise Homogeneous Medium}
Suppose that the spatial domain $U$ of the boundary value problem
for which Maxwell's Equations are to be
solved consists of a disjoint union of a finite number $N$ of
homogeneous regions $U_i$, as in the case of a stratified medium:
\begin{equation}
  U = \union_{i=1}^{N} U_i,
\end{equation}
and suppose that the permittivity and permeability of the $i$th region
are the constants $\epsilon_i$ and $\mu_i$, respectively, with corresponding
wavenumber $k_i$.  Then, for
points within the $i$th medium, the terms involving the gradient of
the permittivity and permeability are zero, and the 
potentials within the $i$th region are solutions to 
\begin{subequations}
  \begin{gather}
    \vlaplace\F^{(i)} + k_i^2 \F^{(i)}  = \epsilon_i\M, \\
    \laplace\Psi^{(i)} + k_i^2 \Psi^{(i)}  = -\qm/\mu_i.
  \end{gather}
\end{subequations}
If the $i$th region contains no sources, then the potentials
in that region are solutions to the Helmholtz equation:
\begin{subequations}
  \begin{gather}
    \vlaplace\F^{(i)} + k_i^2 \F^{(i)}  = \0, \\
    \laplace\Psi^{(i)} + k_i^2 \Psi^{(i)}  = 0.
  \end{gather}
\end{subequations}

\section{Duality}
We note that the cases of electric-only and magnetic-only sources are duals.
Any valid equation involving electromagnetic quantities has a dual equation
which can be obtained by applying the following rules:
\begin{enumerate}
\item Interchange $\mu$ and $\epsilon$.
\item Electric quantities are replaced by their corresponding magnetic quantity.
\item Magnetic quantities are replaced by the negative of their corresponding electric quantity.
\end{enumerate}
The mappings between original and dual quantities are given in 
Table~\ref{tab:duals}.
\begin{table}[htbp]
  \begin{center}
    \leavevmode
    \begin{tabular}{|c|} \hline
      \bfseries Original $\boldsymbol{\rightarrow}$ Dual \\ \hline \hline
      $\mu \rightarrow \epsilon$ \\ \hline
      $\epsilon \rightarrow \mu$ \\ \hline
      $k \rightarrow k$ \\ \hline
      $\eta \rightarrow 1/\eta$ \\ \hline
      $\E \rightarrow \H$ \\ \hline
      $\J \rightarrow \M$ \\ \hline
      $\F \rightarrow \A$ \\ \hline
      $\Phi \rightarrow \Psi$ \\ \hline
      $\qe \rightarrow \qm$ \\ \hline
      $\H \rightarrow -\E$ \\ \hline
      $\M \rightarrow -\J$ \\ \hline
      $\A \rightarrow -\F$ \\ \hline
      $\Psi \rightarrow -\Phi$ \\ \hline
      $\qm \rightarrow -\qe$ \\ \hline
    \end{tabular}
    \caption{Electromagnetic dual quantities.}
    \label{tab:duals}
  \end{center}
\end{table}

\section{Fourier Transform Definitions}
\subsection{One-dimensional Transform}
The Fourier transform of a function $f: \Realnum \rightarrow \Complexnum$
is $\tilde{f} = \Fourier \{f\}$, where
\begin{equation}
  \tilde{f}(k) = \int_{-\infty}^{\infty} f(x) e^{jkx} \d x,
\end{equation}
so that 
\begin{equation}
  f(x) = \frac{1}{2\pi} \int_{-\infty}^{\infty} \tilde{f}(k) e^{-jkx} \d k.
\end{equation}

The completeness statement is
\begin{equation}
  \delta(x) = \frac{1}{2\pi} \int_{-\infty}^{\infty} e^{\pm jkx} \d k
\end{equation}
and Parseval's relation is
\begin{equation}
\int_{-\infty}^{\infty} f(x) g^*(x) \, \d x =   
\frac{1}{2\pi} \int_{-\infty}^{\infty} \tilde{f}(k) \tilde{g}^*(k) \, \d k.
\end{equation}
Finally, if 
\begin{equation}
  h(x) = \int_{-\infty}^{\infty} f(x') g(x-x') \, \d x'
\end{equation}
then the convolution theorem states that
\begin{equation}
  \tilde{h}(k) = \tilde{f}(k) \tilde{g}(k).
\end{equation}


\subsection{Two-dimensional Transform}
The Fourier transform of a function $f: \Realnum \times \Realnum
\rightarrow \Complexnum$
is $\tilde{f} = \Fourier \{f\}$, where
\begin{equation}
  \tilde{f}(k_x,k_y) = 
  \int_{-\infty}^{\infty}\int_{-\infty}^{\infty} 
  f(x,y) e^{j(k_x x + k_y y)} \d x  \d y,
\end{equation}
so that 
\begin{equation}
  f(x,y) = \frac{1}{4\pi^2} 
  \int_{-\infty}^{\infty}\int_{-\infty}^{\infty}
 \tilde{f}(k_x,k_y) e^{-j(k_x x + k_y y)} \d k_x  \d k_y.
\end{equation}

The completeness statement is
\begin{equation}
  \delta(x) \delta(y) = \frac{1}{4\pi^2} 
  \int_{-\infty}^{\infty}\int_{-\infty}^{\infty}
  e^{\pm j(k_x x + k_y y)} \d k_x  \d k_y
\end{equation}
and Parseval's relation is
\begin{equation}
\int_{-\infty}^{\infty} \int_{-\infty}^{\infty}
 f(x,y) g^*(x,y) \, \d x  \d y =   
\frac{1}{4\pi^2} \int_{-\infty}^{\infty} \int_{-\infty}^{\infty} 
\tilde{f}(k_x,k_y) \tilde{g}^*(k_x,k_y) \, \d k_x  \d k_y.
\end{equation}
Finally, if 
\begin{equation}
  h(x,y) = \int_{-\infty}^{\infty}\int_{-\infty}^{\infty}
 f(x',y') g(x-x',y-y') \, \d x'  \d y'
\end{equation}
then the convolution theorem states that
\begin{equation}
  \tilde{h}(k_x,k_y) = \tilde{f}(k_x,k_y) \tilde{g}(k_x,k_y).
\end{equation}


